
\documentclass[journal]{IEEEtran}
\usepackage{amsmath}
\begin{document}

\title{ChemlogSim: the chemical reactions simulator }
\author{Alexandre~Martins,~\IEEEmembership{Member,~IEEE,}
\thanks{Manuscript received April 19, 2005; revised January 11, 2007.}}

\markboth{Journal of \LaTeX\ Class Files,~Vol.~6, No.~1, January~2007}%
{Shell \MakeLowercase{\textit{et al.}}: Bare Demo of IEEEtran.cls for Journals}

\maketitle

\begin{abstract}
The abstract goes here.
\end{abstract}

\begin{IEEEkeywords}
IEEEtran, journal, \LaTeX, paper, template.
\end{IEEEkeywords}

\IEEEpeerreviewmaketitle

\section{Introduction}

\hfill mds
 
\hfill January 11, 2007

\section{Prolog}
The Prolog language includes three different kinds of expressions: facts, rules and queries\cite{merritt1992}.

\section{Euler's method}

\begin{equation}
y'(t) = f(t,y(t)), \qquad \qquad y(t_0)=y_0.
\end{equation}

\section{Resources}
\begin{itemize}
\item SWI-Prolog Manual - http://www.swi-prolog.org/pldoc/index.html
\item Gnuplot 4.2 Manual - http://www.duke.edu/~hpgavin/gnuplot.html
\end{itemize}

\section{Conclusion}
The conclusion goes here.


\ifCLASSOPTIONcaptionsoff
  \newpage
\fi

\bibliography{ZotOutput}
\bibliographystyle{IEEEtran}

\end{document}
