
\documentclass[journal]{IEEEtran}
% Math expressions
\usepackage{amsmath}
% add utf-8 support
\usepackage[utf8]{inputenc}
\begin{document}

\title{ChemlogSim: the chemical reactions simulator }
\author{Alexandre~Martins,~\IEEEmembership{Member,~IEEE,}
\thanks{Manuscript received April 19, 2005; revised January 11, 2007.}}

\markboth{Journal of \LaTeX\ Class Files,~Vol.~6, No.~1, January~2007}%
{Shell \MakeLowercase{\textit{et al.}}: Bare Demo of IEEEtran.cls for Journals}

\maketitle

\begin{abstract}
The abstract goes here.
\end{abstract}

\begin{IEEEkeywords}
IEEEtran, journal, \LaTeX, paper, template.
\end{IEEEkeywords}

\IEEEpeerreviewmaketitle

\section{Introduction}

The main purpose of this work is to buld a chemical reaction simulator in Prolog.  The main raesons to do it within this language and platform are twofold: [i] since the prolog interperter is used, this gives ChemlogSim the hability ,  and (ii) aasdd

\section{Prolog}
The Prolog language includes three different kinds of expressions: facts, rules and queries.

Prolog is much better understood as a language with two unusual features--unification and backtracking. (It also makes heavy use of recursion, which is more common than unification and backtracking but difficult to grasp if you haven't encountered it before.)

Prolog's interface, however (along with much of the Prolog literature), deliberately leads the developer into a conceptual model of logic--and away from Prolog's true inner workings. The syntax of the language is derived from Horn clauses (an area of logic), and early teaching examples emphasize the Prolog-logic connection.\cite{merritt1992}.

[...]

Prolog, billed as "logic programming", is not really. You may be disappointed if that's what you expected to find. On the other hand, having backtracking, unification, and recursion inside one computer language leads to something very powerful and special.\cite{merritt1992}


\section{Euler's method}
\begin{equation}
y'(t) = f(t,y(t)), \qquad \qquad y(t_0)=y_0.
\end{equation}

\section{Resources}
\begin{itemize}
\item SWI-Prolog Manual - http://www.swi-prolog.org/pldoc/index.html
\item Gnuplot 4.2 Manual - http://www.duke.edu/~hpgavin/gnuplot.html
\end{itemize}

\section{Difficulties}

The usage of Prolog as caused me some difficulties that are worth mentioning:
\begin{itemize}
\item Documentation takes too much time with declarative programming without giving a strong base of the procedural part up until later. And a problem that handles calculations is more a procedural problem in nature (ordered steps to reach a goal) then declarative.
\item O coud not find a standard associative array type structure and operations in Prolog. This means handling quirks of specific implementations.
\item There is no clear way of sharing state between different rules. This is a big problem since a system of differential equations is solved numerically by sharing the values of the variables between the equations in different time steps.
\end{itemize}

\section{Conclusion}
The conclusion goes here.


\ifCLASSOPTIONcaptionsoff
  \newpage
\fi

\bibliography{ZotOutput}
\bibliographystyle{IEEEtran}

\end{document}
